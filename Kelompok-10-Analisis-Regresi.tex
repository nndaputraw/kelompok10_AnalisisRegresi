% Options for packages loaded elsewhere
\PassOptionsToPackage{unicode}{hyperref}
\PassOptionsToPackage{hyphens}{url}
%
\documentclass[
]{article}
\usepackage{amsmath,amssymb}
\usepackage{iftex}
\ifPDFTeX
  \usepackage[T1]{fontenc}
  \usepackage[utf8]{inputenc}
  \usepackage{textcomp} % provide euro and other symbols
\else % if luatex or xetex
  \usepackage{unicode-math} % this also loads fontspec
  \defaultfontfeatures{Scale=MatchLowercase}
  \defaultfontfeatures[\rmfamily]{Ligatures=TeX,Scale=1}
\fi
\usepackage{lmodern}
\ifPDFTeX\else
  % xetex/luatex font selection
\fi
% Use upquote if available, for straight quotes in verbatim environments
\IfFileExists{upquote.sty}{\usepackage{upquote}}{}
\IfFileExists{microtype.sty}{% use microtype if available
  \usepackage[]{microtype}
  \UseMicrotypeSet[protrusion]{basicmath} % disable protrusion for tt fonts
}{}
\makeatletter
\@ifundefined{KOMAClassName}{% if non-KOMA class
  \IfFileExists{parskip.sty}{%
    \usepackage{parskip}
  }{% else
    \setlength{\parindent}{0pt}
    \setlength{\parskip}{6pt plus 2pt minus 1pt}}
}{% if KOMA class
  \KOMAoptions{parskip=half}}
\makeatother
\usepackage{xcolor}
\usepackage[margin=1in]{geometry}
\usepackage{color}
\usepackage{fancyvrb}
\newcommand{\VerbBar}{|}
\newcommand{\VERB}{\Verb[commandchars=\\\{\}]}
\DefineVerbatimEnvironment{Highlighting}{Verbatim}{commandchars=\\\{\}}
% Add ',fontsize=\small' for more characters per line
\usepackage{framed}
\definecolor{shadecolor}{RGB}{248,248,248}
\newenvironment{Shaded}{\begin{snugshade}}{\end{snugshade}}
\newcommand{\AlertTok}[1]{\textcolor[rgb]{0.94,0.16,0.16}{#1}}
\newcommand{\AnnotationTok}[1]{\textcolor[rgb]{0.56,0.35,0.01}{\textbf{\textit{#1}}}}
\newcommand{\AttributeTok}[1]{\textcolor[rgb]{0.13,0.29,0.53}{#1}}
\newcommand{\BaseNTok}[1]{\textcolor[rgb]{0.00,0.00,0.81}{#1}}
\newcommand{\BuiltInTok}[1]{#1}
\newcommand{\CharTok}[1]{\textcolor[rgb]{0.31,0.60,0.02}{#1}}
\newcommand{\CommentTok}[1]{\textcolor[rgb]{0.56,0.35,0.01}{\textit{#1}}}
\newcommand{\CommentVarTok}[1]{\textcolor[rgb]{0.56,0.35,0.01}{\textbf{\textit{#1}}}}
\newcommand{\ConstantTok}[1]{\textcolor[rgb]{0.56,0.35,0.01}{#1}}
\newcommand{\ControlFlowTok}[1]{\textcolor[rgb]{0.13,0.29,0.53}{\textbf{#1}}}
\newcommand{\DataTypeTok}[1]{\textcolor[rgb]{0.13,0.29,0.53}{#1}}
\newcommand{\DecValTok}[1]{\textcolor[rgb]{0.00,0.00,0.81}{#1}}
\newcommand{\DocumentationTok}[1]{\textcolor[rgb]{0.56,0.35,0.01}{\textbf{\textit{#1}}}}
\newcommand{\ErrorTok}[1]{\textcolor[rgb]{0.64,0.00,0.00}{\textbf{#1}}}
\newcommand{\ExtensionTok}[1]{#1}
\newcommand{\FloatTok}[1]{\textcolor[rgb]{0.00,0.00,0.81}{#1}}
\newcommand{\FunctionTok}[1]{\textcolor[rgb]{0.13,0.29,0.53}{\textbf{#1}}}
\newcommand{\ImportTok}[1]{#1}
\newcommand{\InformationTok}[1]{\textcolor[rgb]{0.56,0.35,0.01}{\textbf{\textit{#1}}}}
\newcommand{\KeywordTok}[1]{\textcolor[rgb]{0.13,0.29,0.53}{\textbf{#1}}}
\newcommand{\NormalTok}[1]{#1}
\newcommand{\OperatorTok}[1]{\textcolor[rgb]{0.81,0.36,0.00}{\textbf{#1}}}
\newcommand{\OtherTok}[1]{\textcolor[rgb]{0.56,0.35,0.01}{#1}}
\newcommand{\PreprocessorTok}[1]{\textcolor[rgb]{0.56,0.35,0.01}{\textit{#1}}}
\newcommand{\RegionMarkerTok}[1]{#1}
\newcommand{\SpecialCharTok}[1]{\textcolor[rgb]{0.81,0.36,0.00}{\textbf{#1}}}
\newcommand{\SpecialStringTok}[1]{\textcolor[rgb]{0.31,0.60,0.02}{#1}}
\newcommand{\StringTok}[1]{\textcolor[rgb]{0.31,0.60,0.02}{#1}}
\newcommand{\VariableTok}[1]{\textcolor[rgb]{0.00,0.00,0.00}{#1}}
\newcommand{\VerbatimStringTok}[1]{\textcolor[rgb]{0.31,0.60,0.02}{#1}}
\newcommand{\WarningTok}[1]{\textcolor[rgb]{0.56,0.35,0.01}{\textbf{\textit{#1}}}}
\usepackage{graphicx}
\makeatletter
\def\maxwidth{\ifdim\Gin@nat@width>\linewidth\linewidth\else\Gin@nat@width\fi}
\def\maxheight{\ifdim\Gin@nat@height>\textheight\textheight\else\Gin@nat@height\fi}
\makeatother
% Scale images if necessary, so that they will not overflow the page
% margins by default, and it is still possible to overwrite the defaults
% using explicit options in \includegraphics[width, height, ...]{}
\setkeys{Gin}{width=\maxwidth,height=\maxheight,keepaspectratio}
% Set default figure placement to htbp
\makeatletter
\def\fps@figure{htbp}
\makeatother
\setlength{\emergencystretch}{3em} % prevent overfull lines
\providecommand{\tightlist}{%
  \setlength{\itemsep}{0pt}\setlength{\parskip}{0pt}}
\setcounter{secnumdepth}{-\maxdimen} % remove section numbering
\ifLuaTeX
  \usepackage{selnolig}  % disable illegal ligatures
\fi
\IfFileExists{bookmark.sty}{\usepackage{bookmark}}{\usepackage{hyperref}}
\IfFileExists{xurl.sty}{\usepackage{xurl}}{} % add URL line breaks if available
\urlstyle{same}
\hypersetup{
  pdftitle={Kelompok 10 Analisis Regresi},
  pdfauthor={Kelompok 10},
  hidelinks,
  pdfcreator={LaTeX via pandoc}}

\title{Kelompok 10 Analisis Regresi}
\author{Kelompok 10}
\date{2024-02-11}

\begin{document}
\maketitle

install\_tinytex()

Anggota kelompok 9:\\
Ananda Putra Wijaya(G1401221111) Shalma Kaisya Candradewi(G1401221105)
Vito Raditya Pratama(G1401221112)

\begin{Shaded}
\begin{Highlighting}[]
\FunctionTok{library}\NormalTok{(rmarkdown)}
\end{Highlighting}
\end{Shaded}

\begin{verbatim}
## Warning: package 'rmarkdown' was built under R version 4.3.2
\end{verbatim}

\begin{Shaded}
\begin{Highlighting}[]
\FunctionTok{library}\NormalTok{(readxl)}
\end{Highlighting}
\end{Shaded}

\#Membaca data mentah

\begin{Shaded}
\begin{Highlighting}[]
\NormalTok{data }\OtherTok{\textless{}{-}} \FunctionTok{read.csv}\NormalTok{(}\StringTok{"C:/Users/nndap/Downloads/best{-}countries{-}to{-}live{-}in{-}2024.csv"}\NormalTok{)}
\FunctionTok{head}\NormalTok{(data)}
\end{Highlighting}
\end{Shaded}

\begin{verbatim}
##   population_2024 population_growthRate land_area       country        region
## 1      1441719852               0.00916   3287590         India          Asia
## 2      1425178782              -0.00035   9706961         China          Asia
## 3       341814420               0.00535   9372610 United States North America
## 4       279798049               0.00816   1904569     Indonesia          Asia
## 5       245209815               0.01964    881912      Pakistan          Asia
## 6       229152217               0.02389    923768       Nigeria        Africa
##   unMember population_density population_densityMi
## 1     TRUE           484.9067            1255.9084
## 2     TRUE           151.2174             391.6530
## 3     TRUE            37.3673              96.7813
## 4     TRUE           149.0254             385.9758
## 5     TRUE           318.0908             823.8551
## 6     TRUE           251.6027             651.6511
##                                                               share_borders
## 1                                    AFG, BGD, BTN, MMR, CHN, NPL, PAK, LKA
## 2 AFG, BTN, MMR, HKG, IND, KAZ, PRK, KGZ, LAO, MAC, MNG, PAK, RUS, TJK, VNM
## 3                                                                  CAN, MEX
## 4                                                             TLS, MYS, PNG
## 5                                                        AFG, CHN, IND, IRN
## 6                                                        BEN, CMR, TCD, NER
##   Hdi2021 Hdi2020 WorldHappiness2022
## 1   0.633   0.642              3.777
## 2   0.768   0.764              5.585
## 3   0.921   0.920              6.977
## 4   0.705   0.709              5.240
## 5   0.544   0.543              4.516
## 6   0.535   0.535              4.552
\end{verbatim}

\#Mendefinisikan Variabel

\begin{Shaded}
\begin{Highlighting}[]
\NormalTok{Y}\OtherTok{\textless{}{-}}\NormalTok{data}\SpecialCharTok{$}\NormalTok{Hdi2021}
\NormalTok{X}\OtherTok{\textless{}{-}}\NormalTok{data}\SpecialCharTok{$}\NormalTok{population\_growthRate}
\end{Highlighting}
\end{Shaded}

\#Mengubah nama variabel

\begin{Shaded}
\begin{Highlighting}[]
\NormalTok{data}\OtherTok{\textless{}{-}}\FunctionTok{data.frame}\NormalTok{(}\FunctionTok{cbind}\NormalTok{(Y,X))}
\FunctionTok{head}\NormalTok{(data)}
\end{Highlighting}
\end{Shaded}

\begin{verbatim}
##       Y        X
## 1 0.633  0.00916
## 2 0.768 -0.00035
## 3 0.921  0.00535
## 4 0.705  0.00816
## 5 0.544  0.01964
## 6 0.535  0.02389
\end{verbatim}

\#Menghitung jumlah baris dan kolom

\begin{Shaded}
\begin{Highlighting}[]
\NormalTok{n\_baris}\OtherTok{\textless{}{-}}\FunctionTok{nrow}\NormalTok{(data)}
\NormalTok{n\_kolom}\OtherTok{\textless{}{-}}\FunctionTok{ncol}\NormalTok{(data)}
\FunctionTok{nrow}\NormalTok{(data)}
\end{Highlighting}
\end{Shaded}

\begin{verbatim}
## [1] 141
\end{verbatim}

\begin{Shaded}
\begin{Highlighting}[]
\FunctionTok{ncol}\NormalTok{(data)}
\end{Highlighting}
\end{Shaded}

\begin{verbatim}
## [1] 2
\end{verbatim}

Data yang diambil untuk digunakan berasal dari situs Kaggle. Data yang
digunakan merupakan data tentang negara yang enak untuk ditinggali.
Kriteria cocok dilihat berdasarkan nilai Indeks Pembangunan
Manusia(IPM). IPM yang digunakan pada data ini merupakan IPM tahun 2021.

Variabel Y merupakan skor IPM, sedangkan variabel X merupakan angka laju
pertumbuhan penduduk. Kami ingin menganalisis, apakah semakin tinggi
angka laju pertumbuhan penduduk, semakin tinggi pula nilai IPM sebuah
negara.

Nilai IPM berada dalam rentang 0 sampai dengan 1, dengan semakin tinggi
nilai IPM sebuah negara, semakin enak negara tersebut untuk ditinggali.
\#Statistik variabel Y

\begin{Shaded}
\begin{Highlighting}[]
\FunctionTok{summary}\NormalTok{(Y)}
\end{Highlighting}
\end{Shaded}

\begin{verbatim}
##    Min. 1st Qu.  Median    Mean 3rd Qu.    Max. 
##  0.3940  0.6070  0.7540  0.7372  0.8750  0.9620
\end{verbatim}

\#Statistik variabel X

\begin{Shaded}
\begin{Highlighting}[]
\FunctionTok{summary}\NormalTok{(X)}
\end{Highlighting}
\end{Shaded}

\begin{verbatim}
##      Min.   1st Qu.    Median      Mean   3rd Qu.      Max. 
## -0.030870  0.001180  0.008920  0.009327  0.018230  0.038090
\end{verbatim}

\#Scatter plot antara angka laju pertumbuhan penduduk(X) dengan IPM(Y)

\begin{Shaded}
\begin{Highlighting}[]
\FunctionTok{plot}\NormalTok{(X,Y)}
\end{Highlighting}
\end{Shaded}

\includegraphics{Kelompok-10-Analisis-Regresi_files/figure-latex/unnamed-chunk-8-1.pdf}
Berdasarkan Scatter plot, dapat dilihat bahwa hubungan antara X dan Y
merupakan hubungan linear negatif, dimana semakin besar nilai variabel
X, semakin kecil nilai variabel Y

\#Analisis data dengan fungsi LM

\begin{Shaded}
\begin{Highlighting}[]
\NormalTok{model}\OtherTok{\textless{}{-}}\FunctionTok{lm}\NormalTok{(Y}\SpecialCharTok{\textasciitilde{}}\NormalTok{X,data}\OtherTok{\textless{}{-}}\NormalTok{data)}
\FunctionTok{summary}\NormalTok{(model)}
\end{Highlighting}
\end{Shaded}

\begin{verbatim}
## 
## Call:
## lm(formula = Y ~ X, data = data <- data)
## 
## Residuals:
##      Min       1Q   Median       3Q      Max 
## -0.33771 -0.07793 -0.01243  0.07637  0.24132 
## 
## Coefficients:
##             Estimate Std. Error t value Pr(>|t|)    
## (Intercept)  0.82002    0.01175   69.79   <2e-16 ***
## X           -8.88008    0.77032  -11.53   <2e-16 ***
## ---
## Signif. codes:  0 '***' 0.001 '**' 0.01 '*' 0.05 '.' 0.1 ' ' 1
## 
## Residual standard error: 0.1104 on 139 degrees of freedom
## Multiple R-squared:  0.4888, Adjusted R-squared:  0.4851 
## F-statistic: 132.9 on 1 and 139 DF,  p-value: < 2.2e-16
\end{verbatim}

Berdasarkan hasil fungsi LM, diperoleh dugaan persamaan regresi adalah
sebagai berikut \[\hat{y}=0.82002-8.88008{x}\] \#Interpretasi Nilai b0
adalah -8.88008 dan nilai b1 adalah 0.82002. Artinya jika nilai angka
laju pertumbuhan penduduk meningkat 1 satuan, maka dugaan nilai IPM akan
menurun sebesar 8.88008 dan saat angka laju pertumbuhan penduduk
bernilai 0, maka nilai IPM akan bernilai 0.82002. Perlu diingat bahwa
nilai IPM berada pada rentang 0 sampai dengan 1.

\begin{Shaded}
\begin{Highlighting}[]
\NormalTok{(anova.model}\OtherTok{\textless{}{-}}\FunctionTok{anova}\NormalTok{(model))}
\end{Highlighting}
\end{Shaded}

\begin{verbatim}
## Analysis of Variance Table
## 
## Response: Y
##            Df Sum Sq Mean Sq F value    Pr(>F)    
## X           1 1.6195 1.61951  132.89 < 2.2e-16 ***
## Residuals 139 1.6940 0.01219                      
## ---
## Signif. codes:  0 '***' 0.001 '**' 0.01 '*' 0.05 '.' 0.1 ' ' 1
\end{verbatim}

\#Ukuran kebaikan model

\begin{Shaded}
\begin{Highlighting}[]
\NormalTok{(Koef\_det}\OtherTok{\textless{}{-}}\DecValTok{1}\SpecialCharTok{{-}}\NormalTok{(anova.model}\SpecialCharTok{$}\StringTok{\textasciigrave{}}\AttributeTok{Sum Sq}\StringTok{\textasciigrave{}}\NormalTok{[}\DecValTok{2}\NormalTok{]}\SpecialCharTok{/}\FunctionTok{sum}\NormalTok{(anova.model}\SpecialCharTok{$}\StringTok{\textasciigrave{}}\AttributeTok{Sum Sq}\StringTok{\textasciigrave{}}\NormalTok{)))}
\end{Highlighting}
\end{Shaded}

\begin{verbatim}
## [1] 0.4887613
\end{verbatim}

Koefisien determinasi menunjukkan angka sebesar 0.4887613 atau 48.876\%.
Artinya variasi nilai IPM dapat dijelaskan oleh variasi angka laju
pertumbuhan penduduk sebesar 48.876\%. Sisa 51.124\% variasi nilai IPM
dijelaskan oleh faktor atau variabel lain di luar model.

\#Keragaman dugaan parameter

\begin{Shaded}
\begin{Highlighting}[]
\FunctionTok{qt}\NormalTok{(}\FloatTok{0.025}\NormalTok{,}\AttributeTok{df=}\NormalTok{n\_baris}\DecValTok{{-}2}\NormalTok{,}\AttributeTok{lower.tail=}\ConstantTok{FALSE}\NormalTok{)}
\end{Highlighting}
\end{Shaded}

\begin{verbatim}
## [1] 1.977178
\end{verbatim}

\#Dugaan parameter \(\beta_0\)

\begin{Shaded}
\begin{Highlighting}[]
\NormalTok{(b0}\OtherTok{\textless{}{-}}\NormalTok{model}\SpecialCharTok{$}\NormalTok{coefficients[[}\DecValTok{1}\NormalTok{]])}
\end{Highlighting}
\end{Shaded}

\begin{verbatim}
## [1] 0.8200195
\end{verbatim}

\begin{Shaded}
\begin{Highlighting}[]
\NormalTok{(se\_b0}\OtherTok{\textless{}{-}}\FunctionTok{sqrt}\NormalTok{(anova.model}\SpecialCharTok{$}\StringTok{\textasciigrave{}}\AttributeTok{Mean Sq}\StringTok{\textasciigrave{}}\NormalTok{[}\DecValTok{2}\NormalTok{]}\SpecialCharTok{*}\NormalTok{(}\DecValTok{1}\SpecialCharTok{/}\NormalTok{n\_baris}\SpecialCharTok{+}\FunctionTok{mean}\NormalTok{(X)}\SpecialCharTok{\^{}}\DecValTok{2}\SpecialCharTok{/}\FunctionTok{sum}\NormalTok{((X}\SpecialCharTok{{-}}\FunctionTok{mean}\NormalTok{(X))}\SpecialCharTok{\^{}}\DecValTok{2}\NormalTok{))))}
\end{Highlighting}
\end{Shaded}

\begin{verbatim}
## [1] 0.01174944
\end{verbatim}

\begin{Shaded}
\begin{Highlighting}[]
\NormalTok{(t\_b0 }\OtherTok{\textless{}{-}}\NormalTok{ b0}\SpecialCharTok{/}\NormalTok{se\_b0)}
\end{Highlighting}
\end{Shaded}

\begin{verbatim}
## [1] 69.79221
\end{verbatim}

\#Hipotesis Uji \(\beta_0\) \(H_0:\beta_0=0\) (semua nilai IPM dapat
dijelaskan oleh angka laju pertumbuhan penduduk)\\
\(H_1:\beta_0≠0\) (ada nilai IPM yang tidak dapat dijelaskan oleh angka
laju pertumbuhan penduduk)

Karena t hitung(69.79) lebih besar daripada t tabel(1.977), kesimpulan
yang dapat diambil adalah tolak \(H_0\). Ada cukup bukti untuk
menyatakan bahwa ada nilai IPM yang tidak dapat dijelaskan oleh angka
laju pertumbuhan penduduk.

\#Dugaan Parameter \(\beta_1\)

\begin{Shaded}
\begin{Highlighting}[]
\NormalTok{(b1}\OtherTok{\textless{}{-}}\NormalTok{model}\SpecialCharTok{$}\NormalTok{coefficients[[}\DecValTok{2}\NormalTok{]])}
\end{Highlighting}
\end{Shaded}

\begin{verbatim}
## [1] -8.880076
\end{verbatim}

\begin{Shaded}
\begin{Highlighting}[]
\NormalTok{(se\_b1}\OtherTok{\textless{}{-}}\FunctionTok{sqrt}\NormalTok{(anova.model}\SpecialCharTok{$}\StringTok{\textasciigrave{}}\AttributeTok{Mean Sq}\StringTok{\textasciigrave{}}\NormalTok{[}\DecValTok{2}\NormalTok{]}\SpecialCharTok{/}\FunctionTok{sum}\NormalTok{((X}\SpecialCharTok{{-}}\FunctionTok{mean}\NormalTok{(X))}\SpecialCharTok{\^{}}\DecValTok{2}\NormalTok{)))}
\end{Highlighting}
\end{Shaded}

\begin{verbatim}
## [1] 0.7703227
\end{verbatim}

\begin{Shaded}
\begin{Highlighting}[]
\FunctionTok{abs}\NormalTok{((t\_b1}\OtherTok{\textless{}{-}}\NormalTok{b1}\SpecialCharTok{/}\NormalTok{se\_b1))}
\end{Highlighting}
\end{Shaded}

\begin{verbatim}
## [1] 11.52774
\end{verbatim}

\#Hipotesis Uji \(\beta_1\) \(H_0:\beta_1=0\) (tidak ada hubungan linear
antara angka laju pertumbuhan penduduk dengan nilai IPM)
\(H_1:\beta_1≠0\) (ada hubungan linear antara angka laju pertumbuhan
penduduk dengan nilai IPM)

Karena t hitung(11.52) lebih besar daripada t tabel(1.977), kesimpulan
yang dapat diambil adalah tolak \(H_0\). Ada cukup bukti untuk
menyatakan bahwa ada hubungan linear antara angka laju pertumbuhan
penduduk dengan nilai IPM.

\#Selang kepercayaan parameter \(\beta_0\)

\begin{Shaded}
\begin{Highlighting}[]
\NormalTok{(sk.b0}\OtherTok{\textless{}{-}}\FunctionTok{c}\NormalTok{(b0}\SpecialCharTok{{-}}\FunctionTok{abs}\NormalTok{(}\FunctionTok{qt}\NormalTok{(}\FloatTok{0.025}\NormalTok{, }\AttributeTok{df=}\NormalTok{n\_baris}\DecValTok{{-}2}\NormalTok{))}\SpecialCharTok{*}\NormalTok{se\_b0, b0 }\SpecialCharTok{+} \FunctionTok{abs}\NormalTok{(}\FunctionTok{qt}\NormalTok{(}\FloatTok{0.025}\NormalTok{, }\AttributeTok{df=}\NormalTok{n\_baris}\DecValTok{{-}2}\NormalTok{))}\SpecialCharTok{*}\NormalTok{se\_b0))}
\end{Highlighting}
\end{Shaded}

\begin{verbatim}
## [1] 0.7967887 0.8432502
\end{verbatim}

Pada taraf nyata 5\%, diyakini bahwa dugaan parameter \(\beta_0\) berada
dalam selang 0.7765219 sampai 0.8635170

\#Selang kepercayaan \(\beta_1\)

\begin{Shaded}
\begin{Highlighting}[]
\NormalTok{(sk.b1}\OtherTok{\textless{}{-}}\FunctionTok{c}\NormalTok{(b1}\SpecialCharTok{{-}}\FunctionTok{abs}\NormalTok{(}\FunctionTok{qt}\NormalTok{(}\FloatTok{0.025}\NormalTok{, }\AttributeTok{df=}\NormalTok{n\_baris}\DecValTok{{-}2}\NormalTok{))}\SpecialCharTok{*}\NormalTok{se\_b1, b1 }\SpecialCharTok{+} \FunctionTok{abs}\NormalTok{(}\FunctionTok{qt}\NormalTok{(}\FloatTok{0.025}\NormalTok{, }\AttributeTok{df=}\NormalTok{n\_baris}\DecValTok{{-}2}\NormalTok{))}\SpecialCharTok{*}\NormalTok{se\_b1))}
\end{Highlighting}
\end{Shaded}

\begin{verbatim}
## [1] -10.403141  -7.357011
\end{verbatim}

Pada taraf nyata 5\%, diyakini bahwa dugaan parameter \(\beta_1\) berada
dalam selang -8.924083 sampai -8.836069

\#Selang kepercayaan rataan nilai harapan amatan

\begin{Shaded}
\begin{Highlighting}[]
\NormalTok{dugaan\_amatan }\OtherTok{\textless{}{-}} \FunctionTok{data.frame}\NormalTok{(}\AttributeTok{X=}\FloatTok{0.00675}\NormalTok{)}
\FunctionTok{predict}\NormalTok{(model, dugaan\_amatan, }\AttributeTok{interval =} \StringTok{"confidence"}\NormalTok{)}
\end{Highlighting}
\end{Shaded}

\begin{verbatim}
##         fit       lwr       upr
## 1 0.7600789 0.7412831 0.7788748
\end{verbatim}

Sebagai contoh, jika ingin menduga rataan nilai harapan amatan saat
angka laju pertumbuhan penduduk di suatu negara adalah 0.00675,
diperoleh rataan nilai IPM sebesar 0.76 dan dengan taraf kepercayaan
95\%, diyakini bahwa dugaan rataan nilai IPM saat angka laju pertumbuhan
penduduk adalah 0.00675 berada dalam selang 0.7412831 sampai dengan
0.7788748.

\#Selang kepercayaan individu amatan

\begin{Shaded}
\begin{Highlighting}[]
\FunctionTok{predict}\NormalTok{(model, dugaan\_amatan, }\AttributeTok{interval =} \StringTok{"prediction"}\NormalTok{)}
\end{Highlighting}
\end{Shaded}

\begin{verbatim}
##         fit       lwr       upr
## 1 0.7600789 0.5410013 0.9791566
\end{verbatim}

Sebagai contoh, jika ingin menduga nilai individu amatan saat angka laju
pertumbuhan penduduk di suatu negara adalah 0.00675, maka diperoleh
dugaan nilai IPM sebesar 0.76 dan dengan taraf kepercayaan 95\%,
diyakini bahwa nilai amatan individu nilai IPM saat angka laju
pertumbuhan penduduk adalah 0.00675 berada dalam selang 0.5410013 sampai
dengan 0.9791566

\end{document}
